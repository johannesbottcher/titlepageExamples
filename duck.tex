% arara: lualatex
\documentclass{scrbook}
\usepackage{fontspec}
\usepackage{blindtext}
\setmainfont{EBGaramond}
\setsansfont{TeX Gyre Bonum}
\setkomafont{disposition}{\normalfont}
\begin{document}
\begin{titlepage}
	\setlength{\parindent}{0pt}
	\centering  Dixie Dancing Ducks\par
\end{titlepage}
\begin{titlepage}
	%https://github.com/johannesbottcher/titlepageExamples/blob/master/duck.tex
	\centering
	\vspace*{.15\textheight}
	{\Large\addfontfeature{Letters=SmallCaps} Paulo Duck\par}
	\vspace*{.15\textheight}
	{\itshape\Huge Dixie Dancing Ducks\par}
	\vfill
	\centering
	2015\par
	Pond Press Ltd., The Pond
	\vspace{.05\textheight}
\end{titlepage}
	Dedicated to all the ducks in the world\par
	\vfill
	\thispagestyle{empty}
	\begin{flushleft}
		Paulo Duck\\
		Dixie Dancing Ducks\\
		2015\\
		Typeset using LuaLaTeX and EB Garamond\\
	\end{flushleft}
\tableofcontents
\chapter{Introduction}
\blindtext

\blindtext

\blindtext




\setchapterpreamble[u]{
\dictum[Donald]{test me if you can}
}
\chapter{Materials and Methods}
\blindtext {\sffamily Lorem Ipsum} \blindtext

\begin{figure}[b]
	\centering
	\rule{.6\linewidth}{3cm}
\caption{A black rectangle where a picture could be}
\end{figure}
\blindtext

\blindtext[4]
\end{document}
% This example does a very strange thing, it defines a serif-font
% to take place as the default for the sans-serif part. That
% means two serifs are used in one document. Be careful here. 
